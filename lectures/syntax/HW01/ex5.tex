\section{Satzstruktur im Deutschen}
Bestimmen Sie Konstituenten und ihre Kategorien (nur NP, AP, PP, AdvP und CP für eingebettete Sätze) durch Klammerung, Wortarten durch Subskripte, sofern vorhanden die Funktioneekt und Objekt (als zusätzliches Label der Konstituentenklammer) für die folgenden Sätze.
Markieren Sie darüber hinaus für alle Satze die topologischen Felder.


\begin{enumerate}
  \item NP, Subj [Er$_{N}$] hat$_{Verb}$ [NP [den$_{Det}$ Mann$_{N}N$] PP [im$_{Präp}$ Schlafanzug$_{N}$]] begrüßt$_{Verb}$.
  \item AdvP [ Hier$_{Adv}$ [PP[in$_{Prep}$] NP [der$_{Det}$  Stadt$_{N}$]]] hat$_{Verb}$ NP[Paul$_{N}$] AdvP [[ heute$_{Adv}$ ][NP AkkusativObj [das$_{Det}$ NP[große$_{Adj}$ Buch$_{N}$ ]]]] [seiner$_{Pron}$ Freundin$_{N}$] [sehr$_{Adv}$ feierlich$_{Adv}$[ überreicht$_{V}$.
  \item NP, Subj [Regine$_{N}$ hat$_{V}$] NP[die$_{Det}$ Frau$_{N}$] PP [[mit$_{Prep}$ NP [dem$_{Det}$ Fernglas$_{N}$]] gesehen].$_{V}$
  \item CP, Subj CP [[Dass$_{Conj}$] [NP [NP [Fritz$_{N}$] VP [gelacht$_{V}$ hat]]],$_{V}$ bewirkte,$_{V}$ CP [[dass$_{Conj}$ Paul$_{N}$ ging]].$_{V}$
\end{enumerate}
\resizebox{\columnwidth}{!}{%
\begin{tabular}{l|l|l|l}
	Vorfeld & Linke Klammer & Mittelfeld & Rechte Klammer\\ \hline
  Er & hat & den Mann im Schlafanzug & begrüßt. \\ \hline
  Hier in der Stadt & hat & Paul heute das große Buch seiner Freundin sehr feierlich & überreicht. \\ \hline
  Regine & hat & die Frau mit dem Fernglas & gesehen. \\ \hline
  Dass Fritz & gelacht hat, & bewirkte, dass Paul ging. & \\ \hline
\end{tabular}
}