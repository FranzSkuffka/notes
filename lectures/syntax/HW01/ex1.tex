\section{Konstituententests}

Weisen Sie anhand von jeweils min. drei Konstituententests nach, dass
1. die geklammerten Ausdrücke keine Konstituenten bilden:
• Gestern habe ich im Hof [eine Stunde lang ein neues] Auto angesehen.
Fragetest: Wie habe ich im Hof Auto angesehen? - Kein Deutsch.
Permutationstest: [eine Stunde lang ein neues] Gestern habe ich im Hof Auto angesehen. - Was?
Koordinationstest: Gestern habe ich im Hof [eine Stunde lang ein neues] und [eine Stunde lang ein altes] Auto angesehen. - Hmmm? Uneindeutig.
• Neulich haben im Hof [den ganzen Tag vier fröhliche] Kinder gespielt.

2. die geklammerten Ausdrücke eine Konstituente bilden:
• [Gestern] haben im Hof Kinder gespielt.
Eliminierungstest: Haben im Hof Kinder gespielt? \textit{Semantik verändert}
Fragetest: Wann haben die Kinder im Hof gespielt? - Gestern
Koordinationstest: Gestern und Heute haben die Kinder im Hof gespielt.

• Gestern haben im Hof [Kinder] gespielt.
Eliminierungstest: Gestern haben im Hof gespielt? \textit{Subjekt fehlt}
Fragetest: Wer hat gestern im Hof gespielt? - Kinder
Koordinationstest: Gestern und Heute haben die Kinder und Nachbarskinder im Hof gespielt.
Permutationstest: Kinder Gestern und Heute haben die Kinder und Nachbarskinder im Hof gespielt.

3. die geklammerten Ausdrücke eine (zwar) diskontinuierliche Konstituente (aber eben doch eine Konstituente) bilden:
• Ich habe dort [viele Möglichkeiten] gesehen, [die zum Ziel führen].
Permutationstest: Ich habe dort [viele Möglichkeiten], [die zum Ziel führen], gesehen.
Eliminierungstest: Ich habe dort gesehen, [die zum Ziel führen].
Eliminierungstest: Ich habe dort [viele Möglichkeiten] gesehen.
-> Es fällt auf, dass, dass der zweite Teil der Konstituente weggelassen werden kann.
• [Möglichkeiten] wird es [viele] geben, [die zum Ziel führen].
Permutationstest: [Möglichkeiten], [die zum Ziel führen], wird es [viele] geben.
Koordinationstest: [Möglichkeiten] wird es [viele] geben, [die zum Ziel führen].
Fragetest: Was wird es geben? - Viele Möglichkeiten, die zum Ziel führen.