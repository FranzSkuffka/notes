\section{Konstituententests}

Weisen Sie anhand von jeweils min. drei Konstituententests nach, dass

\subsection{die geklammerten Ausdrücke keine Konstituenten bilden:}

Gestern habe ich im Hof [eine Stunde lang ein neues] Auto angesehen.
\begin{itemize}
  \item Fragetest: Wie habe ich im Hof Auto angesehen? - Kein Deutsch.
  \item Permutationstest: [eine Stunde lang ein neues] Gestern habe ich im Hof Auto angesehen. - Was?
  \item Koordinationstest: Gestern habe ich im Hof [eine Stunde lang ein neues] und [eine Stunde lang ein altes] Auto angesehen. - Hmmm? Uneindeutig.
\end{itemize}

Neulich haben im Hof [den ganzen Tag vier fröhliche] Kinder gespielt.


\subsection{die geklammerten Ausdrücke eine Konstituente bilden:}
[Gestern] haben im Hof Kinder gespielt.
\begin{itemize}
  \item Eliminierungstest: Haben im Hof Kinder gespielt? \textit{Semantik verändert}
  \item Fragetest: Wann haben die Kinder im Hof gespielt? - Gestern
  \item Koordinationstest: Gestern und Heute haben die Kinder im Hof gespielt.
\end{itemize}


Gestern haben im Hof [Kinder] gespielt.
\begin{itemize}
  \item Eliminierungstest: Gestern haben im Hof gespielt? \textit{Subjekt fehlt}
  \item Fragetest: Wer hat gestern im Hof gespielt? - Kinder
  \item Koordinationstest: Gestern und Heute haben die Kinder und Nachbarskinder im Hof gespielt.
\end{itemize}
Permutationstest: Kinder Gestern und Heute haben die Kinder und Nachbarskinder im Hof gespielt.

\subsection{die geklammerten Ausdrücke eine (zwar) diskontinuierliche Konstituente (aber eben doch eine Konstituente) bilden:}

Ich habe dort [viele Möglichkeiten] gesehen, [die zum Ziel führen].
\begin{itemize}
  \item Permutationstest: Ich habe dort [viele Möglichkeiten], [die zum Ziel führen], gesehen.
  \item Eliminierungstest: Ich habe dort gesehen, [die zum Ziel führen].
  \item Eliminierungstest: Ich habe dort [viele Möglichkeiten] gesehen.
\end{itemize}
-> Es fällt auf, dass, dass der zweite Teil der Konstituente weggelassen werden kann.

[Möglichkeiten] wird es [viele] geben, [die zum Ziel führen].
\begin{itemize}
  \item Permutationstest: [Möglichkeiten], [die zum Ziel führen], wird es [viele] geben.
  \item Koordinationstest: [Möglichkeiten] wird es [viele] geben, [die zum Ziel führen].
  \item Fragetest: Was wird es geben? - Viele Möglichkeiten, die zum Ziel führen.
\end{itemize}
