\section{Syntaktische Funktionen}
Bestimmen Sie das Subjekt in folgenden Satzen. Welche Kriterien benutzen Sie?
\begin{enumerate}
  \item Der Zug fährt im Bahnhof ein.
  \item Dem Schaffner ist nicht zu helfen.
  \item Dem Schaffner gefällt seine neue Dienstmütze.
  \item ... dass den Kindern Bücher geschenkt wurden.
  \item ... dass Fritz Maria Bücher schenkt.
\end{enumerate}

\begin{enumerate}
  \item Der \textbf{Zug} fährt im Bahnhof ein. \begin{itemize}
  	\item Frage: Wer fährt in den Bahnhof ein?
  	\item Kongruenz: mit finitem Verb `gefällt'
  	\item Kasus: Nominativ
  \end{itemize}

  \item Dem Schaffner ist nicht zu helfen.
  \item Dem Schaffner gefällt seine neue \textbf{Dienstmütze}. \begin{itemize}
  	\item Frage: Was gefällt?
  	\item Kongruenz: mit finitem Verb `Mütze gefällt'
  	\item Kasus: Nominativ `Die Mütze'
  \end{itemize}
  \item ... dass den Kindern \textbf{Bücher} geschenkt wurden. \begin{itemize}
  	\item Frage: Was wird geschenkt? Bücher.
  	\item Kongruenz: mit finitem Verb `Die Bücher werden geschenkt'
  	\item Kasus: Nominativ `Die Bücher'
  \end{itemize}
  \item ... dass \textbf{Fritz} Maria Bücher schenkt. \begin{itemize}
  	\item Frage: Wer schenkt Bücher?
  	\item Kongruenz: mit finitem Verb `Fritz schenkt'
  	\item Kasus: Nominativ
  \end{itemize}
\end{enumerate}