\subsection{Phrasenstruktur}
Geben Sie für die folgenden einfachen Nominalphrasen und Verbalphrasen jeweils eine moglichst generelle Konstituentenstruktur und eine entsprechende Phrasenstrukturregel an. Markieren Sie jeweils den Kopf der Phrase.
\begin{itemize}
  \item NP: die neuen Geschäfte in der Ladenzeile
  \item NP: berühmte Hersteller von Sportkleidung
  \item NP: sehr eingängige Musik
  \item AP: auf seine Schwester eifersüchtig
  \item PP: unter der Brücke
  \item PP: bis zur nächsten Ecke
  \item VP: (weil sie) die Sache klar durchschaut
  \item VP: (dass er) dieses Thema aus seinem Kurs gut kennt
\end{itemize}

\pagebreak
\Tree[.NP [.NP [.Det die ] [.NP [.Adj neuen ] [.N \textbf{Geschäfte} ] ] ] [.PP [.Prep in ] [.NP [.Det der ] [.N Ladenzeile ] ] ] ]
\begin{verbatim}
NP -> Adj N
NP -> Det N
NP -> Det NP
NP -> NP PP
PP -> Prep NP
\end{verbatim}
\vspace{1cm}

\Tree[.NP [.NP [.Adj berühmte ] [.P \textbf{Hersteller} ] ] [.PP [.Prep von ] [.P Sportkleidung ] ] ]
\begin{verbatim}
NP -> Adj P
NP -> NP PP
PP -> Prep P
\end{verbatim}
\vspace{1cm}

\Tree[.NP [.Part sehr ] [.NP [.Adj eingängige ] [.N \textbf{Musik} ] ] ]
\begin{verbatim}
NP -> Adj N
NP -> Part NP
\end{verbatim}
\vspace{1cm}


\Tree[.AP [.PP [.Prep auf ] [.NP [.Pron seine ] [.N Schwester ] ] ] [.Adj \textbf{eifersüchtig} ]]
\begin{verbatim}

AP -> PP Adj
NP -> Pron N
PP -> Prep NP
\end{verbatim}
\Tree[.PP [.Prep \textbf{unter} ] [.NP [.Det der ] [.N Brücke ] ]]
\begin{verbatim}
NP -> Det N
PP -> Prep NP
\end{verbatim}
\vspace{1cm}

\Tree[.PP [.Prep bis ] [.PP [.Prep \textbf{zur} ] [.NP [.Adj nächsten ] [.N Ecke ] ] ] ]
\begin{verbatim}
NP -> Adj N
PP -> Prep NP
PP -> Prep PP
\end{verbatim}
\Tree[.VP [ (weil sie) ] [.VP [.NP [.Det die ] [.N Sache ] ]  [.VP [.Adv klar ] [.V \textbf{durchschaut}  ] ] ] ]
\begin{verbatim}
NP -> Det N
VP -> Adv V
VP -> NP VP
VP -> VP VP
\end{verbatim}
\Tree[.VP [ (dass er) ] [.NP [.NP [.Det dieses ] [.N Thema ] ] [.PP [.Prep aus ] [.NP [.Pron seinem ]  [.N Kurs ] ] ] ] [.VP [.Adv gut ] [.V \textbf{kennt} ]  ] ]
\begin{verbatim}
NP -> Det N
NP -> NP PP
NP -> Pron N
PP -> Prep NP
VP -> Adv V
VP -> NP NP
\end{verbatim}
