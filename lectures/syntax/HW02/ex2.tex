\pagebreak
\section{Die XLE-Web-Grammatik}

\subsection{Passiv}
Die Passivität der Phrase wird in der F-Struktur als Verbtyp dargestellt.
\begin{itemize}
	\item Sie wurde begrüßt.
	\item Brot muss gebacken werden.	
	\item Die Gesellschaft wurde durch einen Schauer aufgescheucht.
\end{itemize}

\subsection{Reflexiv}
Das reflexiv führt ein Objekt ein, welches das Pronomen als Prädikat ausweist.
\begin{itemize}
	\item Fritz rasiert sich nicht.
	\item Maria ängstigt sich.
\end{itemize}

\subsection{Komplement/Adjunkt}
In den Beispielen wird die `Spiegel' als Objekt und `Tag' als Adjunkt geparsed.
Die beiden Phrasen haben eine komplett verschiedene Semantik.
\begin{itemize}
	\item Fritz liest den ganzen Spiegel.
	\item Fritz liest den ganzen Tag.
\end{itemize}


\subsection{Raising/Kontrolle}
Raising drückt sich im Verbtyp aus, das kontrollierte Verb wird zum Parameter des Kontrollierenden Verbs.
\begin{itemize}
	\item Fritz droht die Beziehung zu beenden.
	\item Maria verspricht Hans sich zu bessern.	
	\item Fritz droht zu wenden.
\end{itemize}

% \subsection{Satzeinbettung}

% \begin{itemize}
%	\item Fritz lügt, ohne rot zu werden.
%	\item Wenn Maria ein Fahrrad hätte, wäre sie oft in der Stadt.	
% \end{itemize}

\subsection{Koordination}
Koordinationen werden in der F-Struktur als Menge ausgewiesen. Zudem wird der Koordinationstyp in der umgebenden F-Struktur angegeben.
\begin{itemize}
	\item Fritz und Maria gehen gern ins Kino.
	\item Peter kommt nach Hause und stellt das Radio an.
\end{itemize}



